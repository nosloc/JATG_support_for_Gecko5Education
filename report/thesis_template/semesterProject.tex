%%%%%%%%%%%%%%%%%%%%%%%%%%%%%%%%%%%%%%%%%%%%%%%%%%%%%%%%%%%
% EPFL report package, main thesis file
% Goal: provide formatting for theses and project reports
% Author: Mathias Payer <mathias.payer@epfl.ch>
%
% To avoid any implication, this template is released into the
% public domain / CC0, whatever is most convenient for the author
% using this template.
%
%%%%%%%%%%%%%%%%%%%%%%%%%%%%%%%%%%%%%%%%%%%%%%%%%%%%%%%%%%%
\documentclass[a4paper,11pt,oneside]{report}
% Options: MScThesis, BScThesis, MScProject, BScProject
\usepackage[MScProject,lablogo]{EPFLreport}
\usepackage{xspace}

\title{JTAG support \\for Gecko5Education board}
\author{Antoine Colson}
\supervisor{Prof. Ties Jan Henderikus Kluter}

\newcommand{\boardName}{Gecko5Education \xspace}

\begin{document}
\maketitle
% \makededication
% \makeacks

% \begin{abstract}
% The \sysname tool enables lateral decomposition of a multi-dimensional
% flux compensator along the timing and space axes.

% The abstract serves as an executive summary of your project.
% Your abstract should cover at least the following topics, 1-2 sentences for
% each: what area you are in, the problem you focus on, why existing work is
% insufficient, what the high-level intuition of your work is, maybe a neat
% design or implementation decision, and key results of your evaluation.
% \end{abstract}

% \begin{frenchabstract}
% For a doctoral thesis, you have to provide a French translation of the
% English abstract. For other projects this is optional.
% \end{frenchabstract}

\maketoc

%%%%%%%%%%%%%%%%%%%%%%
\chapter{Introduction}
%%%%%%%%%%%%%%%%%%%%%%

    This project focuses on implementing and testing robust JTAG (Joint Test Action Group) communication 
and control on the \boardName board. The goal is to develop 
low-level infrastructure that enables JTAG-based access to internal system components, 
such as memory and peripherals, paving the way for advanced debugging, testing, and programming capabilities.
Several challenges arose from this project statement.
First, the board was designed exclusively for educational use at EPFL, and as a result, it lacks official documentation which leaves its internals largely unexplored in this context. 
Second, no prior work had been done on enabling JTAG access for the \boardName board; existing references focuse on different boards or architectures. 
Finally, the built-in JTAG interface is extremely minimalistic, exposing only the bare features required by the standard, with only little support for custom extensions or advanced control.\\

To address these challenges, the project was divided into two main milestones: 

\begin{itemize}
    \item Establish basics communication via the JTAG interface and demonstrate control by using 
    on-board RGB LEDs.
    \item Extend the JTA interface to suppoort memory accesses and peripherals control, by providing read and write
    operations to any component on the bus architecture.
\end{itemize}

In summary, this project aimed to transform the previously underutilized JTAG interface of the \boardName board into a powerful low-level communication path
that can later be used by EPFL students. 
Both milestones were successfully completed: reliable communication with the board was achieved, 
and the interface was extended to provide full memory and peripheral access. As a result, 
software can now interact with the board through a high-level abstraction of the JTAG interface, hiding much of the underlying complexity.
The remainder of this report details the background of the project, the design decisions made, the concrete implementation and the evaluation of the updated JTAG interface.


%%%%%%%%%%%%%%%%%%%%
\chapter{Background}
%%%%%%%%%%%%%%%%%%%%

This section introduces the background of the project. It includes introduction on the GECKO5Education board, basic FPGA principles, the board’s initial system architecture, and the JTAG protocol.
It also briefly covers the open-source toolchain used for development, simulation, and debugging.
The background section introduces the necessary background to understand your

\section{The \boardName board}
\label{sec:board}

The \boardname board is a purpose-built FPGA development platform used primarily 
for teaching digital systems design and computer architecture at EPFL.
It is designed to provides strudents a hands-on experience. It features a Lattice ECP5 FPGA at its core. 
But also includes a variety of peripherals and interfaces that are suitabble for system on chip experimetation.
Among the main components, we can find: DRAM memory blocks, GPIOs such as push buttons and switches, a RGB LED matrix,
 and a UART interface for serial ommunication.
The board also provides a JTAG port through the USB-C connector, although until now it was not used for any purpose because of
the minimalistic implementation and the existence of the UART one.


\section{FPGA basics}

As explained in the \ref{sec:board} section, the \boardName board is based on a Lattice ECP5 FPGA.
Unlike CPUs, which are used to execute softare, FPGAs are used to program and reprogram hardware logic. 
They allow the implementation and execution of custom hardware designs, by using a hardware description language (HDL)
 Verilog in this project, that describes the structure and the implementation of the logic. The HDL code is then synthesized,
 placed, and routed to create a bitstream file that can be loaded into the FPGA.
In this project the FPGA is used to improve an existing micro-controller architecture by adding components responsible for JTAG support.


\section{Toolchain}

This project relies on the open-source toolchain \texttt{oss-cad-suite}, which is a collection of tools for FPGA development.
Simulation was performed using Icarus Verilog, while waveform visualization — essential for debugging and verification — was handled by GTKWave. 
The design was synthesized and placed using Yosys and nextpnr-ecp5, then converted into a bitstream using ecppack. 
To load the design onto the FPGA, openFPGALoader was used.
Additionally \texttt{openOCD} (Open On-Chip Debugger) played a crucial role in interacting with the JTAG ports of the board throughout the project.
It provides a telnet-accessible environment to issue low level JTAG commands, enabling direct manipulation of the JTAG interface's state and the corresponding
 input signal.

\section{Initial system architecture}

For the scope of this project we consider the core proccessing unit to be 
an OpenRISC based micro-architecture where the OpenRISC 1000 ISA-based 5 stages pipelined processor is connected to other modules through a 32 bits shared bus architecture. 
The default configuration includes a UART interface component, a memory controller, a camera interface, and other modules less relevant for this project.
All this components are tied together by a shared bus architecture and a bus architecture. Moreover in the initial system architecture the bus architecture 
has the ability to be extended with new components, which will be used in the second milestone of this project.
However this architecture is not taken into account in the first milestone, where the JTAG interface is used to control the RGB LEDs, since they can be directly 
mapped to the GPIOs of the board.

\section{JTAG protocol}

To conclude the background chapter, we will introduce the JTAG protocol, which is the core subject of this project.

The Joint Test Action Group (JTAG) protocol is an industry standard introduced in the 1990s as IEEE Standard 1149.1-1990,
 titled "Standard Test Access Port and Boundary-Scan Architecture."
Originally, it was designed for testing and debugging printed circuit boards using techniques such as boundary-scan testing, 
which allows for checking the interconnections between chips without requiring physical access to the pins.

Nowadays, JTAG continues to be used for testing and debugging, but it has also evolved to support a wider range of applications, 
including in-system programming of flash memory, secure device provisioning, and system-level debugging in embedded and multi-core systems.

The JTAG interface is standardized in IEEE 1149.1 and typically includes four mandatory signals and one optional signal:

\begin{itemize}
\item \textbf{TDI (Test Data In)} – Serial input data line.
\item \textbf{TDO (Test Data Out)} – Serial output data line.
\item \textbf{TCK (Test Clock)} – Clock signal that synchronizes data shifting.
\item \textbf{TMS (Test Mode Select)} – Controls the state of the internal JTAG state machine.
\item \textbf{(Optional) TRST (Test Reset)} – An optional reset line for the JTAG logic.
\end{itemize}

These signals form a serial scan chain that enables test instructions and data to be shifted into and out of the device. 
Internally, JTAG is managed by a TAP (Test Access Port) controller, which uses a finite state machine to control operations like data capture, shifting, and updating.

To better understand the JTAG protocol, it's useful to consider several key states in the TAP state machine:

\begin{itemize}
\item \textbf{Test-Logic Reset (TRST)}: Resets the TAP controller and internal JTAG logic.
\item \textbf{Run-Test/Idle}: A waiting state where no data is being shifted.
\item \textbf{Capture-IR}: Captures data into the instruction register.
\item \textbf{Shift-IR}: Shifts instruction data in and out of the instruction register.
\item \textbf{Update-IR}: Loads new instructions into the instruction register.
\item \textbf{Capture-DR}: Captures data into a selected data register.
\item \textbf{Shift-DR}: Shifts data in and out of the data register.
\item \textbf{Update-DR}: Loads new data into the data register.
\end{itemize}

The TAP controller transitions between these states based on the TMS signal, evaluated on the rising edge of TCK. 
Therefore, by controlling TMS, TCK, and TDI, a user can navigate the TAP state machine, send specific instructions, and shift in data.

Since the JTAG protocol is based on shift registers, the user can also read back data by shifting it out through TDO while new data is shifted in via TDI. 
This enables bidirectional communication for testing, programming, and debugging at a low level and in our case, it allows us to control the LEDs and access the 
memory and peripherals of the \boardName board.
%%%%%%%%%%%%%%%%
\chapter{Design}
%%%%%%%%%%%%%%%%

Introduce and discuss the design decisions that you made during this project.
Highlight why individual decisions are important and/or necessary. Discuss
how the design fits together.

This section is usually 5-10 pages.


%%%%%%%%%%%%%%%%%%%%%%%%
\chapter{Implementation}
%%%%%%%%%%%%%%%%%%%%%%%%

The implementation covers some of the implementation details of your project.
This is not intended to be a low level description of every line of code that
you wrote but covers the implementation aspects of the projects.

This section is usually 3-5 pages.


%%%%%%%%%%%%%%%%%%%%
\chapter{Evaluation}
%%%%%%%%%%%%%%%%%%%%

In the evaluation you convince the reader that your design works as intended.
Describe the evaluation setup, the designed experiments, and how the
experiments showcase the individual points you want to prove.

This section is usually 5-10 pages.


%%%%%%%%%%%%%%%%%%%%%%
\chapter{Related Work}
%%%%%%%%%%%%%%%%%%%%%%

The related work section covers closely related work. Here you can highlight
the related work, how it solved the problem, and why it solved a different
problem. Do not play down the importance of related work, all of these
systems have been published and evaluated! Say what is different and how
you overcome some of the weaknesses of related work by discussing the 
trade-offs. Stay positive!

This section is usually 3-5 pages.


%%%%%%%%%%%%%%%%%%%%
\chapter{Conclusion}
%%%%%%%%%%%%%%%%%%%%

In the conclusion you repeat the main result and finalize the discussion of
your project. Mention the core results and why as well as how your system
advances the status quo.

\cleardoublepage
\phantomsection
\addcontentsline{toc}{chapter}{Bibliography}
\printbibliography

% Appendices are optional
% \appendix
% %%%%%%%%%%%%%%%%%%%%%%%%%%%%%%%%%%%%%%
% \chapter{How to make a transmogrifier}
% %%%%%%%%%%%%%%%%%%%%%%%%%%%%%%%%%%%%%%
%
% In case you ever need an (optional) appendix.
%
% You need the following items:
% \begin{itemize}
% \item A box
% \item Crayons
% \item A self-aware 5-year old
% \end{itemize}

\end{document}